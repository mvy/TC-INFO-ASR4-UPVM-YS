%%%%\%%%%%%%%%%%%%%%%%%%%%%%%%%%%%%%%%%%%%%%%%%%
%% Introduction aux Systèmes d'exploitation  %%
%%   * Historique                            %%
%%   * Principes fondamentaux                %%
%%   * Grandes classes de systèmes           %%
%%%%%%%%%%%%%%%%%%%%%%%%%%%%%%%%%%%%%%%%%%%%%%%

\title{Systèmes d'exploitation, 2ème année}
\subtitle{Système de gestion des fichiers - Partie intermédiaire}

\author{Yves \textsc{Stadler}}\institute{Université Paul Verlaine - Metz}

\date{\today}

\begin{document}


%%
% Page de Titre
%%
\begin{frame}
\titlepage
\end{frame}

\def\sectitle{Agenda}
\section{\sectitle}

\begin{frame}{\sectitle}
\def\subsectitle{Plan du cours}
\subsection{\subsectitle}
\begin{block}{\subsectitle}
\begin{itemize}
    \item Arborescence des fichiers
    \item Liaison decripteur de fichier avec fichier
    \item Montages
\end{itemize}
\end{block}
\end{frame}

\def\sectitle{Liaison avec la partie symbolique}
\subsection{\sectitle}
%%Frame
\begin{frame}[containsverbatim]{\sectitle}
%%Block
\def\subsectitle{Organisation connue}
\subsection{\subsectitle}
\begin{block}{\subsectitle}
\begin{itemize}
    \item Nos fichiers sont utilisables grâce aux descripteurs de fichiers
    \item Nous n'avons toujours pas lié nom de fichier avec le fichier réel
    \item Les fichiers sont organisés en arborescence
    \item Répertoire : collections de fichiers (dont répertoires)
\end{itemize}
\end{block}

\def\subsectitle{Organisation sur le dique}
\subsection{\subsectitle}
\begin{block}{\subsectitle}
    \begin{itemize}
        \item Sous unix un fichier est associé à un numéro \textit{i-node}
        \item La commande \verb+ls -i+ affiche les \textit{i-nodes}
        \item On verra qu'il existe des tables d'\textit{i-nodes}
    \end{itemize}
\end{block}

\end{frame}


\def\sectitle{Modifications de l'arborescence}
\subsection{\sectitle}
%%Frame
\begin{frame}{\sectitle}
%%Block
\def\subsectitle{Liens}
\subsection{\subsectitle}
\begin{block}{\subsectitle}
\begin{itemize}
    \item On peut créer des liens dans l'arborescence
    \item Liens symboliques (un nom associé à un autre)
    \item Liens matériels (plusieurs nom pour une donnée)
\end{itemize}
\end{block}


\def\subsectitle{Montage}
\subsection{\subsectitle}
\begin{block}{\subsectitle}
    \begin{itemize}
        \item Accrocher une arborescence dans une autre arborescence
        \item Abstraction de partitions
        \item Raccourcis, accès confinés.
    \end{itemize}
\end{block}

\end{frame}


%%Frame
\begin{frame}{\sectitle}

\def\subsectitle{Liens symboliques - Soft links}
\subsection{\subsectitle}
\begin{block}{\subsectitle}
    \begin{itemize}
        \item Peut pointer sur quelque-chose d'inexistant
        \item Modifier le contenu du lien revient à modifier le fichier
        \item Enlever le lien ne détruit pas le fichier
        \item \textit{i-nodes} différents.
    \end{itemize}
\end{block}

\def\subsectitle{Commande}
\subsection{\subsectitle}
\begin{exampleblock}{\subsectitle}
    \begin{itemize}
        \item ln -s <cible> [<destination>]
        \item par défaut, répertoire en cours même nom
    \end{itemize}
\end{exampleblock}
\end{frame}


%%Frame
\begin{frame}{\sectitle}
\def\subsectitle{Liens matériels - hard links}
\subsection{\subsectitle}
\begin{block}{\subsectitle}
    \begin{itemize}
        \item Doit pointer sur un fichier existant
        \item Modifier le contenu d'un lien revient à modifier le fichier
        \item Enlever le lien peut détruire le fichiers
        \item Notion de nombre de liens pointant sur le fichier
        \item Un \textit{i-node} unique.
    \end{itemize}
\end{block}


\def\subsectitle{Commande}
\subsection{\subsectitle}
\begin{exampleblock}{\subsectitle}
    \begin{itemize}
        \item ln <cible> [<destination>]
        \item par défaut, répertoire en cours même nom
    \end{itemize}
\end{exampleblock}
\end{frame}


%%Frame
\begin{frame}{\sectitle}
%%Block
\def\subsectitle{Liaison entre abstraction et i-node}
\subsection{\subsectitle}
\begin{block}{\subsectitle}
\begin{itemize}
    \item Une entrée de la table des fichiers ouverts pointe vers un
        \textit{i-node}
    \item Elle contient aussi :
        \begin{itemize}
            \item Le mode d'ouverture
            \item L'offset
        \end{itemize}
\end{itemize}
\end{block}
\end{frame}


\def\sectitle{Fonctions systèmes en C}
\subsection{\sectitle}
%%Frame
\begin{frame}[containsverbatim]{\sectitle}
%%Block
\def\subsectitle{Création, effacement}
\subsection{\subsectitle}
\begin{exampleblock}{\subsectitle}
\begin{itemize}
    \item \verb+link(2)+
    \item \verb+unlink(2)+
    \item \verb+symlink(2)+
    \item \verb+rename(2)+
\end{itemize}
\end{exampleblock}
\end{frame}


\def\sectitle{Répertoires}
\subsection{\sectitle}
%%Frame
\begin{frame}[containsverbatim]{\sectitle}
%%Block
\def\subsectitle{Contenu}
\subsection{\subsectitle}
\begin{block}{\subsectitle}
\begin{itemize}
    \item Les répertoires contiennent des fichiers
    \item En fait, il s'agit d'une liste de (nom, \textit{i-nodes}) 
\end{itemize}
\end{block}

\def\subsectitle{Fonctions systèmes en C}
\subsection{\subsectitle}
\begin{exampleblock}{\subsectitle}
    \begin{itemize}
        \item \verb+mkdir(2)+
        \item \verb+rmdir(2)+
    \end{itemize}
\end{exampleblock}
\end{frame}

%%Frame
\begin{frame}[containsverbatim]{\sectitle}
%%Block
\def\subsectitle{Manipulation des répertoires}
\subsection{\subsectitle}
\begin{exampleblock}{\subsectitle}
    \begin{itemize}
        \item \verb+opendir(3)+
        \item \verb+readdir(3)+
        \item \verb+rewinddir(3)+
        \item \verb+closedir(3)+
    \end{itemize}
\end{exampleblock}

\def\subsectitle{Usage}
\subsection{\subsectitle}
\begin{block}{\subsectitle}
    \begin{itemize}
        \item Une entrée permet de connaître: \textit{i-node}, longueur, type,
            nom et décalage pour l'entrée suivante
        \item On rappelle la fonction read pour accéder à l'entrée suivante
        \item Ces fonctions ne sont pas des appels systèmes. C'est une surcouche
            provenatn d'une bibliothèque.
        \item Les appels systèmes sous-jacents sont
            \verb+open(2),read(2),write(2),close(2)+
    \end{itemize}
\end{block}
\end{frame}


\def\sectitle{Montages}
\subsection{\sectitle}
%%Frame
\begin{frame}[containsverbatim]{\sectitle}
%%Block
\def\subsectitle{Concept}
\subsection{\subsectitle}
\begin{block}{\subsectitle}
\begin{itemize}
    \item Les arboresences peuvent être distinguées en fonctions des supports
        (windows: un lecteur par disque, clef USB, ...)
    \item UNIX introduit la notion de montage, qui efface cette notion de
        support
    \item Un montage est l'association d'un répertoire à une arborescence
    \item Le format de l'arborescence montée peut être différente de celui de
        l'hôte.
    \item On ne peut monter que des volumes.
\end{itemize}
\end{block}

\def\subsectitle{Utilisation}
\subsection{\subsectitle}
\begin{exampleblock}{\subsectitle}
    \begin{itemize}
        \item \verb+mount(8)+ montage par ligne de commande
        \item \verb+mount(2)+ montage par programme.
    \end{itemize}
\end{exampleblock}

\end{frame}



%%Frame
\begin{frame}[containsverbatim]{\sectitle}
%%Block
\def\subsectitle{Montages automatiques}
\subsection{\subsectitle}
\begin{block}{\subsectitle}
\begin{itemize}
    \item Au démarrage de linux, certains montages peuvent être configurés pour
        être automatisés. 
    \item \verb+/etc/fsatb+
\end{itemize}
\end{block}


\def\subsectitle{Exemple}
\subsection{\subsectitle}
\begin{exampleblock}{\subsectitle}
\begin{verbatim}
# <file system> <mount point>   <type>  <options>       <dump>  <pass>
proc            /proc           proc    nodev,noexec,nosuid 0       0
# / was on /dev/sda4 during installation 
UUID=1d0c6490-24c8-4af6-a37c-9605c7546774 / ext4 errors=remount-ro 0 1
\end{verbatim}
\end{exampleblock}

\end{frame}

\end{document}
