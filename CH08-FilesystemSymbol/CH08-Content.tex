%%%%\%%%%%%%%%%%%%%%%%%%%%%%%%%%%%%%%%%%%%%%%%%%
%% Introduction aux Systèmes d'exploitation  %%
%%   * Historique                            %%
%%   * Principes fondamentaux                %%
%%   * Grandes classes de systèmes           %%
%%%%%%%%%%%%%%%%%%%%%%%%%%%%%%%%%%%%%%%%%%%%%%%

\title{Systèmes d'exploitation, 2ème année}
\subtitle{Système de gestion de fichiers - Symbolique programmeur}

\author{Yves \textsc{Stadler}}\institute{Université Paul Verlaine - Metz}

\date{\today}

\begin{document}


%%
% Page de Titre
%%
\begin{frame}
\titlepage
\end{frame}

\def\sectitle{Agenda}
\section{\sectitle}
\def\subsectitle{Plan du cours}
\subsection{\subsectitle}

\begin{frame}{\sectitle}
\begin{block}{\subsectitle}
\begin{itemize}
\item Pourquoi les fichiers 
\item Un système de gestion
\item Représentation symbolique externe
\item Représentation symbolique noyau
\end{itemize}
\end{block}
\end{frame}


\def\sectitle{Objectifs et motivations}
\section{\sectitle}
\begin{frame}{\sectitle}
\def\subsectitle{Motivation}
\subsection{\subsectitle}
\begin{block}{\subsectitle}
\begin{itemize}
    \item La mémoire vive disparait après usage
    \item Mémoire vive non persistante (données à durée de vie = vie du processus)
    \item Faible capacité
    \item Non partageable en dehors de l'ordinateur
    \item Aucune organisation personnelle
\end{itemize}
\end{block}

\def\subsectitle{Objectifs}
\subsection{\subsectitle}
\begin{block}{\subsectitle}
\begin{itemize}
    \item Mémoire persistante
    \item Très grande capacités
    \item Partage possible
    \item Organisation des données
\end{itemize}
\end{block}

\end{frame}



\def\sectitle{Le Système de Gestion des Fichiers}
\section{\sectitle}
\begin{frame}{\sectitle}
\def\subsectitle{À quoi sert le SGF}
\subsection{\subsectitle}
\begin{block}{\subsectitle}
\begin{itemize}
    \item Définition d'un fichier
    \item Manipulation d'un fichier
    \item Gérer l'organisation logique
    \item Lien entre le symbolique et le physique
    \item Gestion méta-logique (sécurité, robustesse, ...)
\end{itemize}
\end{block}
\end{frame}


\def\sectitle{Représentation symbolique externe}
\section{\sectitle}
\begin{frame}{\sectitle}
\begin{block}{\subsectitle}
\def\subsectitle{Qu'est-ce qu'un fichier}
\subsection{\subsectitle}
\begin{itemize}
    \item Flux linéaire d'octets
    \item Le programmeur n'a besoin que d'un nombre minimun d'information
    \begin{itemize}
        \item nom
        \item taille
        \item type
        \item propriétaire
        \item dates
        \item protections
    \end{itemize}
\end{itemize}
\end{block}

\def\subsectitle{Flux linéaire d'octets}
\subsection{\subsectitle}
\begin{exampleblock}{\subsectitle}
\begin{itemize}
    \item Tubes
    \item Sockets
    \item Tout ce qui peut entrer dans \texttt{open(2)}
\end{itemize}
\end{exampleblock}
\end{frame}


\begin{frame}{\sectitle}
\def\subsectitle{Format de fichiers}

\def\subsectitle{}
\subsection{\subsectitle}

\def\subsectitle{Contenus}
\subsection{\subsectitle}
\begin{exampleblock}{\subsectitle}
\begin{itemize}
    \item Sources, traitement de texte, ...
    \item Scripts, exécutables, ...
    \item Spéciaux: fichier null, périphérique
    \item Répertoires
\end{itemize}
\end{exampleblock}

\def\subsectitle{Formats}
\subsection{\subsectitle}
\begin{block}{\subsectitle}
\begin{itemize}
    \item Fichiers ordinaires non-exécutables / exécutable
    \item Fichiers spéciaux
    \item Répertoires
\end{itemize}
\end{block}

\def\subsectitle{Attention}
\subsection{\subsectitle}
\begin{alertblock}{\subsectitle}
\begin{itemize}
    \item L'extension est juste un visuel et n'a rien a voir avec le format!
\end{itemize}
\end{alertblock}

\end{frame}

\def\subsectitle{Protections}
\subsection{\subsectitle}
\begin{frame}{\sectitle}
\begin{block}{Champs de protection: ls -l}
    \begin{center}
        \begin{tabular}{|c|c|c|}
        \hline
        U & G & O\\
        \hline
        r x w & r x w & r x w \\
        \hline
        \end{tabular}
    \end{center}

    \begin{itemize}
        \item \texttt{-rwxrwxrwx}
    \end{itemize}
\end{block}

\begin{block}{Légende}
\begin{itemize}
\item  User Group Others
\item  Read Write eXecute
\item  Absence de droit : -
\end{itemize}
\end{block}

\end{frame}

\begin{frame}[containsverbatim]{\sectitle}
\def\subsectitle{Distinction fichier répertoire}
\subsection{\subsectitle}
\begin{block}{\subsectitle}
\begin{itemize}
    \item \texttt{-rwxrwxrwx} : fichier
    \item \texttt{drwxrwxrwx} : répertoire
\end{itemize}
\end{block}

\def\subsectitle{Distinction du format de fichier}
\subsection{\subsectitle}
\begin{block}{\subsectitle}
\begin{verbatim}
ystadler@stage-habbas:~$ file lock
lock: ASCII text
ystadler@stage-habbas:~$ file vimconf.tar.gz
vimconf.tar.gz: gzip compressed data, from Unix, last 
modified: Fri Oct 21 13:49:32 2011
ystadler@stage-habbas:~$
\end{verbatim}
\end{block} 

\end{frame}

\begin{frame}{\sectitle}
\def\subsectitle{Les permissions en C}
\subsection{\subsectitle}
\begin{exampleblock}{\subsectitle}
\begin{itemize}
    \item \texttt{S\_IRUSR S\_IWUSR S\_IXUSR}
    \item \texttt{S\_IRGRP S\_IWGRP S\_IXGRP}
    \item \texttt{S\_IROTH S\_IWOTH S\_IXOTH}
\end{itemize}
\end{exampleblock}

\def\subsectitle{Consultation et modification des droits}
\subsection{\subsectitle}
\begin{block}{\subsectitle}
\begin{itemize}
    \item Appel système \texttt{access(2)} permet de lire ces informations
    \item Appel système \texttt{chmod(2)} permet de modifier ces informations
    \item Il faut que le progamme ait les droits de le faire.
\end{itemize}
\end{block}
\end{frame}



\def\sectitle{Représentation symolique interne}
\section{\sectitle}
\begin{frame}{\sectitle}
\def\subsectitle{Associer le nom de fichier à un élément système}
\subsection{\subsectitle}
\begin{block}{\subsectitle}
\begin{itemize}
    \item Un fichier c'est un emplacement de début sur le disque et une taille
    \item Comment dire que "readme.txt" est à cet endroit
\end{itemize}
\end{block}

\def\subsectitle{Philosophie}
\subsection{\subsectitle}
\begin{block}{\subsectitle}
\begin{itemize}
    \item On a pas besoin d'avoir en permanence un lien vers tous les fichiers
    \item Notion de fichiers ouverts et fermés (\texttt{open(2) close(2)})
    \item On doit conserver un lien avec les fichiers ouverts (efficacité)
    \item Pourquoi ne pas numéroter les fichiers ouverts?
    \item On utilise une table des fichiers ouverts
    \item C'est un cache, recherche physique effectuée une seule fois.
\end{itemize}
\end{block}

\end{frame}


\begin{frame}{\sectitle}
\begin{block}{\subsectitle}
\begin{itemize}
    \item Chaque fichiers dispose d'un numéro d'entrée dans la table de fichiers
    ouverts
    \item Ce numéro est le descripteur de ce fichiers
    \item La table gère ensuite les permissions, liens physique, ...
    \item On stocke dans cette table la position de lecture dans le fichier
\end{itemize}
\end{block}

\begin{exampleblock}{\subsectitle}
\begin{itemize}
    \item \texttt{STDIN\_FILENO} vaut toujours 0 pour tous les processus?
\end{itemize}
\end{exampleblock}

\begin{block}{\subsectitle}
\begin{itemize} 
    \item Table à deux niveaux
    \item Chaque processus gère ses descripteurs de fichiers (dont
    \texttt{STDIN\_FILENO} ...
    \item Chaque élément de la table des fichiers ouverts pointent sur un
    élément de la table des fichiers ouverts du système
    \item On ne peut pas partager de descripteur de fichier entre processus
    (sauf avec fork car duplication).
\end{itemize}
\end{block}
\end{frame}


\end{document}
