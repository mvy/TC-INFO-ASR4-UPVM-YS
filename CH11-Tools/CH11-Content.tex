%%%%\%%%%%%%%%%%%%%%%%%%%%%%%%%%%%%%%%%%%%%%%%%%
%% Introduction aux Systèmes d'exploitation  %%
%%   * Historique                            %%
%%   * Principes fondamentaux                %%
%%   * Grandes classes de systèmes           %%
%%%%%%%%%%%%%%%%%%%%%%%%%%%%%%%%%%%%%%%%%%%%%%%

\title{Systèmes d'exploitation, 2ème année}
\subtitle{Outils utiles}

\author{Yves \textsc{Stadler}}\institute{Université Paul Verlaine - Metz}

\date{\today}

\begin{document}


%%
% Page de Titre
%%
\begin{frame}
\titlepage
\end{frame}

\def\sectitle{Agenda}
\section{\sectitle}
\def\subsectitle{Plan du cours}
\subsection{\subsectitle}

\begin{frame}{\sectitle}
\begin{block}{\subsectitle}
\begin{itemize}
    \item Éditeurs de textes
    \item Gestion des sources
    \item Utiliser POSIX sur windows
    \item Utiliser des versions natives des outils GNU sur windows
    \item Rédaction de texte avec \LaTeX
\end{itemize}
\end{block}
\end{frame}

\def\sectitle{Éditeurs de textes}
\section{\sectitle}
\def\subsectitle{Liste}
\subsection{\subsectitle}
\begin{frame}{\sectitle}
    \begin{block}{\subsectitle}
        \begin{itemize}
            \item Emacs (GUI, debugger, Lisp, metakeys, très personnalisable)
            \item (g)Vim (Text, léger, modal, home row)
            \item Pico
            \item Nano
            \item Eclipse (Différentes versions selon les langages, mise-à-jour
                à la volée, debugger)
            \item Notepad++
        \end{itemize}
    \end{block}

\def\subsectitle{Choix}
\subsection{\subsectitle}
    \begin{block}{\subsectitle}
        \begin{itemize}
            \item Fonctionnalités, adaptation, environnement de travail, choix
                personnel.
        \end{itemize}
    \end{block}
\end{frame}


\def\sectitle{POSIX sous windows}
\section{\sectitle}
\def\subsectitle{Unix avec Cygwin}
\subsection{\subsectitle}
\begin{frame}{\sectitle}
    \begin{block}{\subsectitle}
        \begin{itemize}
            \item Apporte les fonctions POSIX comme fork, signals
            \item Collection d'outils
            \item cygwin1.dll : couche API Unix pour windows
            \item Implémente POSIX en terme d'appel Win32
        \end{itemize}
    \end{block}

\def\subsectitle{Cygwin n'est pas}
\subsection{\subsectitle}
    \begin{block}{\subsectitle}
        \begin{itemize}
            \item Ne permet pas de compiler sous unix et exécuter sur windows
            \item Ne permet pas de rendre les applications natives windows au
                courant des fonctions Unix.
        \end{itemize}
    \end{block}
\end{frame}


\def\sectitle{GNU sous windows}
\section{\sectitle}
\def\subsectitle{MinGW}
\subsection{\subsectitle}
\begin{frame}{\sectitle}
    \begin{block}{\subsectitle}
        \begin{itemize}
            \item Minimalist GNU for Windows
            \item Collection d'outils GNU (sed, awk, gcc, ...)
            \item indépendance vis-à-vis de DLL C propriétaire (dépent cependant
                de DLL windows standard)
        \end{itemize}
    \end{block}

\def\subsectitle{MinGW n'est pas}
\subsection{\subsectitle}
    \begin{block}{\subsectitle}
        \begin{itemize}
            \item N'apporte pas une implémentation complète de POSIX
        \end{itemize}
    \end{block}
\end{frame}


\def\sectitle{Construction de programmes}
\section{\sectitle}
\def\subsectitle{}
\subsection{\subsectitle}
\begin{frame}{\sectitle}
    \begin{block}{\subsectitle}
        \begin{itemize}
            \item Make
            \item ANT (multi-plateforme, basé sur XML pour java au départ)
            \item Autotools, génération des scripts standards utilisés dans les
                projets GNU (./configure, make)
            \item Cmake (multi-plateforme, multiples outils: make, xcode, visual
                studio)
        \end{itemize}
    \end{block}
\end{frame}


\def\sectitle{Debugger/Profile}
\section{\sectitle}
\def\subsectitle{Debuggeur}
\subsection{\subsectitle}
\begin{frame}{\sectitle}
    \begin{block}{\subsectitle}
        \begin{itemize}
            \item GDB (GNU debuger)
            \item Insight (GUI GDB)
            \item DDD (GUI GDB).
        \end{itemize}
    \end{block}


\def\subsectitle{Profileur}
\subsection{\subsectitle}
    \begin{block}{\subsectitle}
        \begin{itemize}
            \item strace (suivi des appels système)
            \item ltrace (tracer les appels de librairies dynamiques)
            \item gprof (GNU profiler, analyse le temps passé dans les
                fonctions)
        \end{itemize}
    \end{block}

\end{frame}

\def\sectitle{Édition de textes avec \LaTeX}
\section{\sectitle}
\def\subsectitle{What you see is what you mean}
\subsection{\subsectitle}
\begin{frame}{\sectitle}
    \begin{block}{\subsectitle}
        \begin{itemize}
            \item Distinction totale entre le fond et la forme
            \item Un fichier .tex décrit le fond (contenu du document) et la
                structure (je suis un titre de chapitre, je suis un document de
                type livre, je suis un paragraphe)
            \item L'entête d'un document \LaTeX permet d'importer les outils de
                forme (typographie française/anglaise/..., recto ou recto/verso,
                colonnes, ...)
            \item Écriture mathématique codifiée et très puissante.
        \end{itemize}
    \end{block}


\def\subsectitle{Analogie avec la programmation}
\subsection{\subsectitle}
    \begin{block}{\subsectitle}
        \begin{itemize}
            \item Le fichier .tex est la source 
            \item Il est compilé avec un compilateur de documents, exemple
                pdflatex qui génère des documents PDFs.
        \end{itemize}
    \end{block}
\end{frame}


\def\subsectitle{Travailler avec \LaTeX}
\subsection{\subsectitle}
\begin{frame}{\sectitle}
    \begin{block}{\subsectitle}
        \begin{itemize}
            \item Un éditeur de texte
            \item Une distribution \LaTeX (Live-TeX, MikTeX, ...)
        \end{itemize}
    \end{block}


\def\subsectitle{Pourquoi \LaTeX plutôt que Word}
\subsection{\subsectitle}
    \begin{block}{\subsectitle}
        \begin{itemize}
            \item Word fait ce que \LaTeX fait en grande partie
            \item Word "cache" un certains nombre de concepts de forme
                (typographie, césures, gestion de styles) que \LaTeX oblige à
                utiliser (comportement par défaut).
            \item Les comportements par défaut sont là parce que la rédaction
                d'un document correct suit des règles de style précis.
        \end{itemize}
    \end{block}
\end{frame}


\def\sectitle{Contrôle de sources}
\section{\sectitle}
\def\subsectitle{Git}
\subsection{\subsectitle}
\begin{frame}{\sectitle}
    \begin{block}{\subsectitle}
        \begin{itemize}
            \item Système distribué de contrôle et révision de source.
            \item Ne repose pas sur un serveur central (voir svn)
            \item Travailler à plusieurs sur de grands projets
        \end{itemize}
    \end{block}


\def\subsectitle{SVN}
\subsection{\subsectitle}
    \begin{block}{\subsectitle}
        \begin{itemize}
            \item Système centralisé de contrôle de sources
            \item Dérivé de CVS
        \end{itemize}
    \end{block}
\end{frame}


\def\sectitle{Auto-documentation}
\section{\sectitle}
\def\subsectitle{Documentation}
\subsection{\subsectitle}
\begin{frame}{\sectitle}
    \begin{block}{\subsectitle}
        \begin{itemize}
            \item Javadoc: documentation pour les projets Java
            \item Doxygen: documentation pour les projets C++
        \end{itemize}
    \end{block}
\end{frame}


\def\sectitle{Ne pas oublier les outils standards}
\section{\sectitle}
\def\subsectitle{Manipulation de texte}
\subsection{\subsectitle}
\begin{frame}{\sectitle}
    \begin{block}{\subsectitle}
        \begin{itemize}
            \item sed, awk
            \item Expression rationelle
            \item Scripts (bash, zsh, ...)
        \end{itemize}
    \end{block}
\end{frame}


\def\sectitle{Aide}
\section{\sectitle}
\def\subsectitle{Beacoup de sources d'aide avec internet}
\subsection{\subsectitle}
\begin{frame}{\sectitle}
    \begin{block}{\subsectitle}
        \begin{itemize}
            \item Man
            \item UseNet (serveur de news) en déclin? ou pas... 
            \item Bibliothèques
            \item Moteurs de recherches, forums
            \item Site de l'éditeur
            \item Stackoverflow (et sites StackExchange)
            \item Collègues
        \end{itemize}
    \end{block}
\end{frame}

\end{document}
