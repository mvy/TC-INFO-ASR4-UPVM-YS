%%%%\%%%%%%%%%%%%%%%%%%%%%%%%%%%%%%%%%%%%%%%%%%%
%% Introduction aux Systèmes d'exploitation  %%
%%   * Historique                            %%
%%   * Principes fondamentaux                %%
%%   * Grandes classes de systèmes           %%
%%%%%%%%%%%%%%%%%%%%%%%%%%%%%%%%%%%%%%%%%%%%%%%

\title{Systèmes d'exploitation, 2ème année}
\subtitle{Primitives Execs}

\author{Yves \textsc{Stadler}}\institute{Université Paul Verlaine - Metz}

\date{\today}

\begin{document}


%%
% Page de Titre
%%
\begin{frame}
\titlepage
\end{frame}

\def\sectitle{Agenda}
\section{\sectitle}
\def\subsectitle{Plan du cours}
\subsection{\subsectitle}

\begin{frame}{\sectitle}
\begin{block}{\subsectitle}
\begin{itemize}
\item Concept
\item Fonctions
\item Réalisation
\end{itemize}
\end{block}
\end{frame}

\def\subsectitle{}
\subsection{\subsectitle}

\def\sectitle{Exec}
\section{\sectitle}
\begin{frame}{\sectitle}
\def\subsectitle{Notion}
\subsection{\subsectitle}
\begin{block}{\subsectitle}
\begin{itemize}
\item Lancer un programme existant depuis un autre programme
\item Idée: remplacer l'image mémoire du processus par une autre
\item Les primitives exec vont nous permettre de faire ça.
\end{itemize}
\end{block}

\def\subsectitle{Remarque}
\subsection{\subsectitle}
\begin{alertblock}{\subsectitle}
\begin{itemize}
\item Attention, lorsque l'on appel une primitive exec celle-ci ne se terminera
pas.
\item Un processus qui appel une primitive exec ne changera pas d'identité.
\end{itemize}

\end{alertblock}

\end{frame}

\begin{frame}[containsverbatim]{\sectitle}
\def\subsectitle{Parmètres}
\subsection{\subsectitle}
\begin{block}{\subsectitle}
\begin{itemize}
    \item Un programme
    \item Des paramètres
    \item Des variables d'environnement.
\end{itemize}
\end{block}

\def\subsectitle{Example}
\subsection{\subsectitle}
\begin{exampleblock}{\subsectitle}
\begin{verbatim}{\subsectitle}
int main(int argc, char * argv[], char * arge[]);
\end{verbatim}
\end{exampleblock}
\end{frame}



\begin{frame}[containsverbatim]{\sectitle}
\def\subsectitle{man exec}
\subsection{\subsectitle}
\begin{block}{\subsectitle}
\begin{verbatim}
NAME
       execl, execlp, execle, execv, execvp - execute a file

SYNOPSIS
       #include <unistd.h>

       extern char **environ;

       int execl(const char *path, const char *arg, ...);
       int execlp(const char *file, const char *arg, ...);
       int execle(const char *path, const char *arg,
                  ..., char * const envp[]);
       int execv(const char *path, char *const argv[]);
       int execvp(const char *file, char *const argv[]);
\end{verbatim}
\end{block}
\end{frame}

\begin{frame}{\sectitle}
\def\subsectitle{Différences}
\subsection{\subsectitle}
\begin{block}{\subsectitle}
\begin{itemize}
    \item Recherche du programme : 
    \begin{itemize}
        \item vide: répertoire en cours (execl, execv, execle)
        \item p : path (execlp, execvp)
    \end{itemize}
    \item Arguments du programme :
    \begin{itemize}
        \item vide: tableau (terminé par NULL)
        \item l : liste explicite (terminée par NULL)
    \end{itemize}
    \item Utiliser:
    \begin{itemize}
        \item vide: environnement courant
        \item e : environnement sous forme de tableau (terminé par NULL)
    \end{itemize}
\end{itemize}
\end{block}
\end{frame}

\begin{frame}{\sectitle}
\def\subsectitle{Fonctionnement}
\subsection{\subsectitle}
\begin{block}{\subsectitle}
\begin{itemize}
    \item Rechercher le fichier pointer par arg1
    \item Vérifier que le fichier est exécutable (permissions)
    \item Vérifier dans l'entête que le fichier peut être chargé
    \item Copier les paramètres de l'exec courant dans l'espace système
    \item Détacher toutes les régions du programme
    \item Allouer de nouvelles régions
    \item Attacher les nouvelles régions
    \item Charger la région en mémoire
    \item Copier les paramètres de l'exec dans la pile utilisateur.
\end{itemize}
\end{block}
\end{frame}

\def\sectitle{}
\section{\sectitle}
\begin{frame}{\sectitle}
\def\subsectitle{Ne pas oublier}
\subsection{\subsectitle}
\begin{block}{\subsectitle}
\begin{itemize}
    \item Sauf si il échoue à se lancer, on ne revient pas d'un exec
    \item Tout le code écrit après un exec qui a réussi n'existe plus!
    (écrasement)
\end{itemize}
\end{block}

\def\subsectitle{Contrôle}
\subsection{\subsectitle}
\begin{block}{\subsectitle}
\begin{itemize}
    \item \texttt{pid\_t wait(int *status);}
    \item \texttt{pid\_t waitpid(pid\_t pid, int *status, int options);}
    \item \texttt{ waitpid(-1, \&status, 0);}
    \item Comment exploiter la variable status.
\end{itemize}
\end{block}

\end{frame}

\def\sectitle{Fin de processus}
\section{\sectitle}
\begin{frame}{\sectitle}
\begin{block}{\subsectitle}
\def\subsectitle{Mort du processus}
\subsection{\subsectitle}
\begin{itemize}
    \item Fonction de tests
    \item WIFEXITED(status) : indique si un processus s'est terminé correctement
    \item WEXITSTATUS(status): code d'erreur retourné par le fils (exit/return)
    \item WIFSIGNALED(status): vrai si fils terminé par un signal
    \item WTERMSIG(satus): renvoi le numéro du signal qui a terminé le fils
\end{itemize}
\end{block}
\end{frame}


\begin{frame}{\sectitle}
\def\subsectitle{Communication intrasèque}
\subsection{\subsectitle}
\begin{block}{\subsectitle}
\begin{itemize}
    \item Comment communiquer avec les fils ?
    \item Redirection d'entrée standard et de sortie standard (dup)
    \item Utilisation des pipes nommés
    \item Utilisation de popen (pipe + fork + shell)
\end{itemize}
\end{block}
\end{frame}

\def\sectitle{Divers}
\section{\sectitle}
\begin{frame}{\sectitle}
\def\subsectitle{Divers}
\subsection{\subsectitle}
\begin{block}{\subsectitle}
\begin{itemize}
    \item \texttt{int system(const char *command);} si on veut juste exécuter
    une commande.
    \item \texttt{vfork}: système dédié nécessitant performance -> père bloqué
    tant que le fils existe et n'a pas fait un \texttt{exevce}
    \item Dans le cas du vfork, il y a un partage de mémoire temporaire. A
    utiliser avec extrème précaution.
    \item Si on veut travailler avec des mémoires partagées, il existe
    \texttt{clone}
    \item Clone est utilisé pour créer des threads.
\end{itemize}
\end{block}
\end{frame}

\end{document}
