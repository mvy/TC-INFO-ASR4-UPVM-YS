%%%%\%%%%%%%%%%%%%%%%%%%%%%%%%%%%%%%%%%%%%%%%%%%
%% Introduction aux Systèmes d'exploitation  %%
%%   * Historique                            %%
%%   * Principes fondamentaux                %%
%%   * Grandes classes de systèmes           %%
%%%%%%%%%%%%%%%%%%%%%%%%%%%%%%%%%%%%%%%%%%%%%%%

\title{Systèmes d'exploitation, 2ème année}
\subtitle{Communication}

\author{Yves \textsc{Stadler}}\institute{Université Paul Verlaine - Metz}

\date{\today}

\begin{document}


%%
% Page de Titre
%%
\begin{frame}
\titlepage
\end{frame}

\def\sectitle{Agenda}
\section{\sectitle}
\def\subsectitle{Plan du cours}
\subsection{\subsectitle}

\begin{frame}{\sectitle}
\def\subsectitle{Plan du cours}
\subsection{\subsectitle}
\begin{block}{\subsectitle}
\begin{itemize}
\item Tubes de communication
\item Tubes nommés
\item Partage de mémoire
\item Signaux
\end{itemize}
\end{block}
\end{frame}

\def\sectitle{Motivations}
\section{\sectitle}
\begin{frame}{\sectitle}
\def\subsectitle{Contexte}
\subsection{\subsectitle}

\begin{exampleblock}{\subsectitle}

\begin{itemize}
    \item Exécution de \texttt{ls -l | cut -f 2 -d ' '}
    \item Création de deux processus concurrents
    \item \texttt{ls} écrit des données dans un `pipe` 
    \item \texttt{cut} lit des données dans un `pipe`
    \item \texttt{ls} termine par un `EOF`
    \item \texttt{cut} lit jusqu'à `EOF`
\end{itemize}
\end{exampleblock}
\def\subsectitle{En programmation}
\subsection{\subsectitle}
\begin{block}{\subsectitle}
\begin{itemize}
    \item Dans le cas d'un pipe système: automatique
    \item Peut-on faire pareil en programmant ?
    \item Avoir un flux qui bloque tant que rien n'est a lire
    \item Pouvoir écrire dans ce flux avec un processus.
\end{itemize}
\end{block}
\end{frame}

\def\sectitle{Tubes}
\section{\sectitle}
\begin{frame}{\sectitle}
\def\subsectitle{Définition}
\subsection{\subsectitle}
\begin{block}{\subsectitle}
\begin{itemize}
    \item Le tube est une file d'attente FIFO
    \item Unidirectionel
    \item Communication par flot
    \item Auto-synchronisé
    \item Appartient au système (comme les sémaphores)
\end{itemize}
\end{block}

\def\subsectitle{Comportement}
\subsection{\subsectitle}
\begin{block}{\subsectitle}
\begin{itemize}
    \item Table des fichiers ouverts (descripteur)
    \item La lecture est destructrice, tout ce qui est lu disparait du tube
    \item Capacité finie, un tube peut être saturé.
\end{itemize}
\end{block}

\end{frame}


\begin{frame}{\sectitle}
\def\subsectitle{Reprrésentation}
\subsection{\subsectitle}
\begin{block}{\subsectitle}
\begin{itemize}
    \item Identifié par un numéro \texttt{i-node} comme un fichier
    \item N'existe pas dans le système de fichier
    \item On utilise les blocs adressés directement
    \item File circulaires (double pointeurs)
\end{itemize}
\end{block}

\def\subsectitle{Aller plus loin}
\subsection{\subsectitle}
\begin{block}{\subsectitle}
\begin{itemize}
    \item On peut aussi nommer un tube
    \item On peut communiquer entre processus étrangers.
\end{itemize}
\end{block}

\end{frame}


\def\sectitle{Fonction pipe}
\section{\sectitle}
\begin{frame}[containsverbatim]{\sectitle}
\begin{block}{\subsectitle}
\begin{verbatim}
NAME
       pipe, pipe2 - create pipe

SYNOPSIS
       #include <unistd.h>

       int pipe(int pipefd[2]);
\end{verbatim}
\end{block}
\end{frame}

\begin{frame}{\sectitle}
\begin{block}{\subsectitle}
\begin{itemize}
    \item Créer un pipe unidirectionnel 
    \item pipefd et rempli avec deux descripteurs
    \item \texttt{[0]} est la sortie du tube, on peut lire \texttt{pipefd[0]}
    \item \texttt{[1]} est l'entrée du tube, on peut écrire \texttt{pipefd[1]}
    \item Le processus qui ne lit pas fermera \texttt{pipefd[0]} (close)
    \item Le processus qui n'écrit pas fermera \texttt{pipefd[1]} (close)
\end{itemize}
\end{block}
\end{frame}

\def\sectitle{Lire et écrire}
\section{\sectitle}
\begin{frame}[containsverbatim]{\sectitle}
\def\subsectitle{man 2 read}
\subsection{\subsectitle}
\begin{block}{\subsectitle}
\begin{verbatim}
NAME
       read - read from a file descriptor

SYNOPSIS
       #include <unistd.h>

       ssize_t read(int fd, void *buf, size_t count);

\end{verbatim}
\end{block}
\end{frame}

\begin{frame}{\sectitle}
\def\subsectitle{Paramètres}
\subsection{\subsectitle}
\begin{block}{\subsectitle}
\begin{itemize}
    \item fd: descripteur de fichier
    \item buf: variable pour recevoir les données
    \item count: nombre d'octets à lire
    \item Attention, read renvoi le nombre d'octets effectivement lus (pas
    forcément ce que l'on a demandé)
\end{itemize}
\end{block}
\end{frame}


\begin{frame}[containsverbatim]{\sectitle}
\def\subsectitle{man 2 write}
\subsection{\subsectitle}
\begin{block}{\subsectitle}
\begin{verbatim}
NAME
       write - write to a file descriptor

SYNOPSIS
       #include <unistd.h>

       ssize_t write(int fd, const void *buf, size_t count);
\end{verbatim}
\end{block}
\end{frame}

\begin{frame}[containsverbatim]{\sectitle}
\def\subsectitle{Paramètres}
\subsection{\subsectitle}
\begin{block}{\subsectitle}
\begin{itemize}
    \item fd: descripteur de fichier
    \item buf: variable source les données
    \item count: nombre d'octets à écrire
    \item Attention, write renvoi le nombre d'octets effectivement écrits (pas
    forcément ce que l'on a demandé)
\end{itemize}
\end{block}
\end{frame}

\def\sectitle{Tubes nommés}
\section{\sectitle}
\begin{frame}[containsverbatim]{\sectitle}
\def\subsectitle{man 3 mkfifo}
\subsection{\subsectitle}
\begin{block}{\subsectitle}
\begin{verbatim}
NAME
       mkfifo - make a FIFO special file (a named pipe)

SYNOPSIS
       #include <sys/types.h>
       #include <sys/stat.h>

       int mkfifo(const char *pathname, mode_t mode);
\end{verbatim}
\end{block}
\end{frame}

\def\sectitle{Opérations sur les descipteurs}
\section{\sectitle}
\begin{frame}{\sectitle}
\def\subsectitle{Dup}
\subsection{\subsectitle}
\begin{block}{\subsectitle}
\begin{itemize}
    \item \texttt{int desc2 = dup (int desc1);}
    \item Duplique une entrée de la table des descripteurs
    \item \texttt{int dup2(int src, int dest);}
    \item Copie dans dest les informations de src
    \item Exemple, remplacer stdin par un tube
\end{itemize}
\end{block}
\end{frame}

\def\sectitle{Mémoire partagée}
\section{\sectitle}
\begin{frame}[containsverbatim]{\sectitle}
\def\subsectitle{Utilisation}
\subsection{\subsectitle}
\begin{block}{\subsectitle}
\begin{itemize}
    \item Similaire à l'utilisation des files et sémaphores
    \item Besoin d'une clef (man ftok), ou IPC\_PRIVATE
    \item Dispose d'une taille
\end{itemize}
\end{block}

\def\subsectitle{man 2 shmget}
\subsection{\subsectitle}
\begin{block} {\subsectitle}
\begin{verbatim}
NAME
       shmget - allocates a shared memory segment

SYNOPSIS
       #include <sys/ipc.h>
       #include <sys/shm.h>

       int shmget(key_t key, size_t size, int shmflg);
\end{verbatim}
\end{block} 
\end{frame}

\def\sectitle{Controle des segments}
\section{\sectitle}
\begin{frame}[containsverbatim]{\sectitle}
\def\subsectitle{man 2 shmctl}
\subsection{\subsectitle}
\begin{block}{\subsectitle}
\begin{verbatim}
NAME
       shmctl - shared memory control

SYNOPSIS
       #include <sys/ipc.h>
       #include <sys/shm.h>

       int shmctl(int shmid, int cmd, struct shmid_ds *buf);

\end{verbatim}
\end{block}

\def\subsectitle{Notes}
\subsection{\sectitle}
\begin{block}{\subsectitle}
\begin{itemize}
    \item Peu utile en général
    \item Lié à la structure shmid\_ds décrite dans le man
\end{itemize}
\end{block}
\end{frame}

\def\sectitle{Utilisation}
\section{\sectitle}
\begin{frame}[containsverbatim]{\sectitle}
\def\subsectitle{man shmat, man shmdt}
\subsection{\subsectitle}
\begin{block}{\subsectitle}
\begin{verbatim}

NAME
       shmat, shmdt - shared memory operations

SYNOPSIS
       #include <sys/types.h>
       #include <sys/shm.h>

       void *shmat(int shmid, const void *shmaddr, int shmflg);

       int shmdt(const void *shmaddr);
\end{verbatim}
\end{block}
\end{frame}

\begin{frame}{\sectitle}
\def\subsectitle{Attacher, détacher}
\subsection{\subsectitle}
\begin{block}{\subsectitle}
\begin{itemize}
    \item \texttt{shmat} Attache shmid dans l'espace d'adressage du processus appelant
    \item shmaddr == NULL: cherche une adresse compatible.
    \item Si shmaddr n'est pas NULL, utiliser SHM\_RND ou faire attention
    d'avoir une adresse alignée avec la page mémoire.
    \item renvoie l'adresse du segment attaché.
    \item \texttt{shmdt} détache shmid.
\end{itemize}
\end{block}
\end{frame}

\def\sectitle{Signaux}
\section{\sectitle}
\begin{frame}{\sectitle}
\def\subsectitle{Interruptions}
\subsection{\subsectitle}
\begin{block}{\subsectitle}
\begin{itemize}
    \item Interruptions logicielles
    \item Réagir à des événements sans continuellement attendre.
    \item Un processus peut dire au système ce qui doit se passer sur un signal.
    \begin{itemize}
        \item Ignorer
        \item Prendre en compte : exécuter une fonction (handler)
        \item Appliquer le comportement par défaut (KILL)
    \end{itemize}
    \item tous les signaux ne sont pas interceptable ni ignorable.
\end{itemize}
\end{block}
\end{frame}


\begin{frame}[containsverbatim]{\sectitle}
\def\subsectitle{man 7 signals}
\subsection{\subsectitle}
\begin{block}{\subsectitle}
\begin{verbatim}
man 7 signal
Signal Value Action Comment
-----------------------------------------------------------
SIGHUP    1   Term  Hangup detected on controlling terminal
                    or death of controlling process
SIGINT    2   Term  Interrupt from keyboard
SIGQUIT   3   Core  Quit from keyboard
SIGILL    4   Core  Illegal Instruction
SIGABRT   6   Core  Abort signal from abort(3)
SIGFPE    8   Core  Floating point exception
SIGKILL   9   Term  Kill signal
...
\end{verbatim}
\end{block}
\end{frame}


\begin{frame}[containsverbatim]{\sectitle}
\def\subsectitle{Signal kill}
\subsection{\subsectitle}
\begin{block}{\subsectitle}
\begin{itemize}
    \item \texttt{int kill(pid\_t pid, int sig);}
    \item Génère/Envoi un signal KILL pour pid
\end{itemize}
\end{block}

\def\subsectitle{Gestion de signaux}
\subsection{\subsectitle}
\begin{block}{\subsectitle}
\begin{verbatim}
NAME
       sigaction - examine and change a signal action

SYNOPSIS
       #include <signal.h>

       int sigaction(int signum, const struct sigaction *act,
                     struct sigaction *oldact);
\end{verbatim}
\end{block}

\end{frame}


\begin{frame}[containsverbatim]{\sectitle}
\def\subsectitle{Structure de gestion des signaux}
\subsection{\subsectitle}
\begin{verbatim}
           struct sigaction {
               void     (*sa_handler)(int);
               void     (*sa_sigaction)(int, siginfo_t *, void *);
               sigset_t   sa_mask;
               int        sa_flags;
               void     (*sa_restorer)(void);
           };
\end{verbatim}
\end{frame}

\def\sectitle{Masquage}
\section{\sectitle}
\begin{frame}{\sectitle}
\def\subsectitle{Masquage}
\subsection{\subsectitle}
\begin{block}{\subsectitle}
\begin{itemize}
    \item Les processus dispose d'un masque qui défini l'ensemble des signaux
    bloqués.
    \item \texttt{sigprocmask} gère cet ensemble
    \item sigemptyset, sigfillset, sigaddset, ...
\end{itemize}
\end{block}

\def\subsectitle{Attente d'un signal}
\subsection{\subsectitle}
\begin{block}{\subsectitle}
\begin{itemize}
    \item pause()
    \item alarm()
\end{itemize}
\end{block}

\end{frame}

\end{document}
