%%%%\%%%%%%%%%%%%%%%%%%%%%%%%%%%%%%%%%%%%%%%%%%%
%% Introduction aux Systèmes d'exploitation  %%
%%   * Historique                            %%
%%   * Principes fondamentaux                %%
%%   * Grandes classes de systèmes           %%
%%%%%%%%%%%%%%%%%%%%%%%%%%%%%%%%%%%%%%%%%%%%%%%

\title{Systèmes d'exploitation, 2ème année}
\subtitle{Fichiers et processus}

\author{Yves \textsc{Stadler}}\institute{Université Paul Verlaine - Metz}

\date{\today}

\begin{document}


%%
% Page de Titre
%%
\begin{frame}
\titlepage
\end{frame}

\def\sectitle{Agenda}
\section{\sectitle}
\def\subsectitle{Plan du cours}
\subsection{\subsectitle}

\begin{frame}{\sectitle}
\begin{block}{\subsectitle}
\begin{itemize}
\item Système de gestion des fichiers
\item Gestion des processus
\item Gestion de la mémoire
\item Communication TCP/IP
\end{itemize}
\end{block}
\end{frame}

\def\sectitle{Introduction}
\section{\sectitle}
\def\subsectitle{Organisation d'un système d'exploitation}
\subsection{\subsectitle}


\begin{frame}{\sectitle}
\begin{block}{\subsectitle}
\begin{itemize}
    \item Chaque système est basé sur un noyau (\textit{kernel});
    \item La noyau comprend deux parties : indépendante du matériel, dépendante
    du matériel;
\end{itemize}
\end{block}

\def\subsectitle{Partie dépendante}
\subsection{\subsectitle}
\begin{block}{\subsectitle}
\begin{itemize}
    \item gestion des interruptions;
    \item gestion mémoire;
    \item gestion des entrées sorties (E/S; IO).
\end{itemize}

\end{block}

\begin{block}{Partie indépendante}
\begin{itemize}
    \item ordonnanceur-distributeur;
    \item gestion des processus;
    \item paginiation, va-et-vient;
    \item sous-système de fichier;
    \item gestion des entrées sorties (partie "haute").
\end{itemize}

\end{block}

\end{frame}


\end{document}
