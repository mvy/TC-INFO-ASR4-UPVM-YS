%%%%\%%%%%%%%%%%%%%%%%%%%%%%%%%%%%%%%%%%%%%%%%%%
%% Introduction aux Systèmes d'exploitation  %%
%%   * Historique                            %%
%%   * Principes fondamentaux                %%
%%   * Grandes classes de systèmes           %%
%%%%%%%%%%%%%%%%%%%%%%%%%%%%%%%%%%%%%%%%%%%%%%%

\title{Systèmes d'exploitation, 2ème année}
\subtitle{Système de gestion de fichiers - Partie physique}

\author{Yves \textsc{Stadler}}\institute{Université Paul Verlaine - Metz}

\date{\today}

\begin{document}


%%
% Page de Titre
%%
\begin{frame}
\titlepage
\end{frame}

\def\sectitle{Agenda}
\section{\sectitle}
\def\subsectitle{Plan du cours}
\section{\subsectitle}

\begin{frame}{\sectitle}
\begin{block}{\subsectitle}
\begin{itemize}
\item Organisation logique du disque: partitions
\item Organisation physique du disque: blocs
\item Démarrage
\end{itemize}
\end{block}
\end{frame}


\def\sectitle{Organisation du disque}
\section{\sectitle}
\begin{frame}{\sectitle}

\def\subsectitle{Subdivisions}
\subsection{\subsectitle}
\begin{block}{\subsectitle}
\begin{itemize}
    \item Disque dur composé de secteurs (512 - 16384 bytes)
    \item de cylindres (20 - 200 secteurs) (pistes situés à même distance de
        l'axe)
    \item de plateaux (800 - 4000 cylindres)
\end{itemize}
\end{block}


\def\subsectitle{Implications}
\subsection{\subsectitle}
\begin{block}{\subsectitle}
    \begin{itemize}
        \item Pour optimiser les transferts, on utilise des blocs
        \item En général, le bloc est de 1KiB
    \end{itemize}
\end{block}
\end{frame}


\def\sectitle{Stockage de données}
\section{\sectitle}
%%Frame
\begin{frame}{\sectitle}
%%Block
\def\subsectitle{Deux méthodes}
\subsection{\subsectitle}
\begin{block}{\subsectitle}
\begin{itemize}
    \item Stockage d'un fichier:
        \begin{itemize}
            \item Blocs contigus
            \item Blocs discontigus
        \end{itemize}
\end{itemize}
\end{block}


\def\subsectitle{Avantages/Inconvénients}
\subsection{\subsectitle}
\begin{block}{\subsectitle}
    \begin{itemize}
        \item Positionnemenent et lecture rapide
        \item Problème de mise à jour quand le bloc est plein et le suivant
            occupé
        \item Problème de la fragmentation lorsque l'on a de l'espace pour
            stocker un fichier, mais que celui-ci n'est pas continu.
    \end{itemize}
\end{block}
\end{frame}


\def\sectitle{Allocations}
\section{\sectitle}
%%Frame
\begin{frame}{\sectitle}
%%Block
\def\subsectitle{Allocation chaînée}
\subsection{\subsectitle}
\begin{block}{\subsectitle}
\begin{itemize}
    \item Pas de fragmentation externe
    \item Pas de limite de taille
    \item Accès direct moins efficace
    \item Problème de perte de pointeur
\end{itemize}
\end{block}
\end{frame}


%%Frame
\begin{frame}{\sectitle}
%%Block
\def\subsectitle{Allocation indexée}
\subsection{\subsectitle}
\begin{block}{\subsectitle}
\begin{itemize}
    \item Chaque fichier possède un bloc d'index (1er bloc)
    \item Un enregistrement dans un répertoire pointe sur le bloc d'index
    \item La ième entrée du bloc d'index pointe sur le ième bloc du fichier
\end{itemize}
\end{block}


\def\subsectitle{Propriétés}
\subsection{\subsectitle}
\begin{block}{\subsectitle}
    \begin{itemize}
        \item Taille de la table des index (i-nodes) proportionnelles aux
            nombres de fichiers
        \item Fragmentation interne
        \item Problème de la taille des index
    \end{itemize}
\end{block}
\end{frame}


%%Frame
\begin{frame}{\sectitle}
%%Block
\def\subsectitle{Problèmes généraux}
\subsection{\subsectitle}
\begin{block}{\subsectitle}
\begin{itemize}
    \item Un bloc ne peut contenir qu'un seul fichier.
    \item Si le fichier est plus petit que la taille d'un bloc, alors on créer
        de la fragmentation interne.
    \item La fragmentation interne est du à un usage inférieur à l'allocation
        demandée/donnée.
\end{itemize}
\end{block}


\def\subsectitle{Fragmentations}
\subsection{\subsectitle}
\begin{block}{\subsectitle}
    \begin{itemize}
        \item Si l'on rempli un espace de 4 unités avec 4 bloc de taille une
            unité, puis on retire la première et la dernière unité, on se
            retrouve avec 2 unités de libres
        \item Si on souhaite remplir l'espace libre avec un bloc de taille 2
            unités, c'est impossible car on ne dispose pas de cet espace
            continu.
        \item La fragmentation externe est l'impossiblité d'utiliser l'espace
            libre, car il est dispersé en blocs trop petits.
    \end{itemize}
\end{block}
\end{frame}


\def\sectitle{Grands fichiers}
\subsection{\sectitle}
%%Frame
\begin{frame}{\sectitle}
%%Block
\def\subsectitle{Extension du bloc}
\subsection{\subsectitle}
\begin{block}{\subsectitle}
\begin{itemize}
    \item Un fichiers sur plusieurs blocs: on a besoin d'un lien
    \item Réserver le dernier mot pour pointer sur le bloc suivant
    \item Multiniveaux: Indexs pointant sur dans indexs pointant sur des blocs
    \item Combiné: Les premières entrées pointent sur des blocs de données, puis
        sur des blocs mutlti-niveaux. (UNIX)
\end{itemize}
\end{block}
\end{frame}

\def\sectitle{EXT3}
\section{\sectitle}
%%Frame
\begin{frame}{\sectitle}
%%Block
\def\subsectitle{Format ext3 d'UNIX}
\subsection{\subsectitle}
\begin{block}{\subsectitle}
\begin{itemize}
    \item Indexé : i-nodes
    \item Taille des blocs dépends du microprocessuer utilisé
    \item Maximum bloc de 8KiB
    \item Taille de la partition maximale : 32TiB si bloc de 8KiB, 16TiB pour
        bloc de 4KiB, 2TiB si bloc de 1KiB
    \item Taille maximale de fichier: 2TiB si 8KiB, 2TiB si 4KiB, 16GiB si 1KiB
\end{itemize}
\end{block}
\end{frame}


\def\sectitle{FAT}
\section{\sectitle}
%%Frame
\begin{frame}{\sectitle}
%%Block
\def\subsectitle{Format FAT de windows}
\subsection{\subsectitle}
\begin{block}{\subsectitle}
\begin{itemize}
    \item On parle de cluster en général
    \item Les clusters sont indexés par un nombre de 32 bits.
    \item Le secteur de boot utilise un champ de 32 bits pour indexer les
        secteurs => 2TiB de 512B ou 16TiB de 4096B
\end{itemize}
\end{block}
\end{frame}


\def\sectitle{}
\section{\sectitle}
%%Frame
\begin{frame}{\sectitle}
%%Block
\def\subsectitle{FAT allocation table}
\subsection{\subsectitle}
\begin{block}{\subsectitle}
\begin{itemize}
    \item Liste d'entrée qui associe à chaque cluster (groupe de secteurs):
        \begin{itemize}
            \item Le prochain cluster
            \item La fin de la chaîne
            \item Une marque de cluster corrompu
            \item Une marque indiquant que le cluster n'est pas utilisé.
        \end{itemize}
    \item Table indexée par numéros de blocs
    \item Une entrée de répertoire contient un pointeur sur le premier bloc.
\end{itemize}
\end{block}
\end{frame}


\def\sectitle{Secteurs spéciaux}
\section{\sectitle}
%%Frame
\begin{frame}{\sectitle}
%%Block
\def\subsectitle{Le premier secteur - secteur de boot}
\subsection{\subsectitle}
\begin{block}{\subsectitle}
\begin{itemize}
    \item Non partitionné (ou dans une partition): secteur de boot = 1er secteur (volume boot record)
    \item Partitionné: secteur de boot = MBR.
    \item Le MBR défini les partitions. Le premier secteur de la partion est en
        général le secteur de boot du format (FAT32 par exemple)
    \item Au démarrage, le PC charge le MBR et l'exécute
\end{itemize}
\end{block}
\end{frame}




%%Frame
\begin{frame}{\sectitle}
%%Block
\def\subsectitle{BPB (FAT, NTFS)}
\subsection{\subsectitle}
\begin{block}{\subsectitle}
\begin{itemize}
    \item BPB (Bios Parameter Block) se situe dans le secteur de boot.
    \item Donne des informations comme le nombre d'octets par secteurs
    \item Nombre de secteur par cluster
    \item Type de disque
    \item Taille de la FAT
\end{itemize}
\end{block}
\end{frame}


\def\sectitle{Divers}
\section{\sectitle}
%%Frame
\begin{frame}[containsverbatim]{\sectitle}
%%Block
\def\subsectitle{Vérifications des disques}
\subsection{\subsectitle}
\begin{block}{\subsectitle}
\begin{itemize}
    \item Vérification des disques: \verb+chkdsk+ \verb+fsck+
    \item Un inodes ne peut pas être libre et faire partie d'un fichier
    \item Parcours des répertoires, vérfication du nombre de liens.
    \item ...
\end{itemize}
\end{block}
\end{frame}

\end{document}
