\title{Systèmes d'exploitation, 2ème année}
\subtitle{Méthodes de synchronisation}

\author{Yves \textsc{Stadler}}\institute{Université Paul Verlaine - Metz}

\date{\today}

\begin{document}


%%
% Page de Titre
%%

%%%%%%%%%%%%%%%%%%%%%%%%%%%%%%%%%%%%%%%%%%%%%%%%%%%%%%%%%%%%%%%%%%%%%%%%%%%%%%%%
\begin{frame}
\titlepage
\end{frame}

\def\sectitle{Agenda}
\section{\sectitle}
\def\subsectitle{Plan du cours}
\subsection{\subsectitle}

%%%%%%%%%%%%%%%%%%%%%%%%%%%%%%%%%%%%%%%%%%%%%%%%%%%%%%%%%%%%%%%%%%%%%%%%%%%%%%%%
\begin{frame}{\sectitle}
    \begin{block}{\subsectitle}
        \begin{itemize}
            \item Pourquoi la synchronisation
            \item Problèmes de synchronisation
            \item Solutions de synchronisation
        \end{itemize}
    \end{block}
\end{frame}


%%%%%%%%%%%%%%%%%%%%%%%%%%%%%%%%%%%%%%%%%%%%%%%%%%%%%%%%%%%%%%%%%%%%%%%%%%%%%%%%
\def\sectitle{Objectifs}
\section{\sectitle}
\def\subsectitle{Pourquoi synchroniser}
\subsection{\subsectitle}
\begin{frame}{\sectitle}
    \begin{block}{\subsectitle}
        \begin{itemize}
            \item Hors du cadre temps réel
            \item Les ressources sont limités et ne peuvent/doivent pas être
                utilisées en même temps;
            \item Éviter l'instabilité du système/programme résultant d'un
                mauvais usage des ressources partagées.
        \end{itemize}
    \end{block}
\end{frame}

%%%%%%%%%%%%%%%%%%%%%%%%%%%%%%%%%%%%%%%%%%%%%%%%%%%%%%%%%%%%%%%%%%%%%%%%%%%%%%%%
\def\sectitle{Race conditions}
\section{\sectitle}
\begin{frame}[containsverbatim]{\sectitle}
    \begin{block}{\subsectitle}
        \begin{itemize}
            \item L'ordonnancement peut arrêter un processus pendant son
                exécution;
            \item Processus plus prioritaire, quantum expiré, ...
        \end{itemize}
    \end{block}

    \begin{columns}[b]

        \column{.5\textwidth}
        \def\subsectitle{Processus 1}
        \subsection{\subsectitle}
        \begin{exampleblock}{\subsectitle}
            \begin{verbatim}
                x++; 
                y = x;
            \end{verbatim}
        \end{exampleblock}

        \column{.5\textwidth}

        \def\subsectitle{Processus 2}
        \subsection{\subsectitle}
        \begin{exampleblock}{\subsectitle}
            \begin{verbatim}
                x++; 
                z = x;
            \end{verbatim}
        \end{exampleblock}

    \end{columns}

    \def\subsectitle{Quel résultat}
    \subsection{\subsectitle}
    \begin{block}{\subsectitle}
        \begin{itemize}
            \item Selon l'ordre d'exécution, les résultats seront différents.
        \end{itemize}
    \end{block}

\end{frame}

%%%%%%%%%%%%%%%%%%%%%%%%%%%%%%%%%%%%%%%%%%%%%%%%%%%%%%%%%%%%%%%%%%%%%%%%%%%%%%%%
\def\sectitle{Exécutions}
\section{\sectitle}
\begin{frame}[containsverbatim]{\sectitle}

\begin{columns}[t]
\column{.5\textwidth}
\def\subsectitle{P1 avant P2}
\subsection{\subsectitle}

\begin{exampleblock}{\subsectitle}
\begin{verbatim}
P1: x++; // x == 1
P1: y = x; // y == 1
P2: x++; // x == 2
P2: z = x; // z == 2
\end{verbatim}
\end{exampleblock}

\def\subsectitle{Incrément en premier}
\subsection{\subsectitle}

\begin{exampleblock}{\subsectitle}
\begin{verbatim}
P1: x++; // x == 1
P2: x++; // x == 2
P1: y = x; // y == 2
P2: z = x; // z == 2
\end{verbatim}
\end{exampleblock}

\column{.5\textwidth}
\def\subsectitle{P2 avant P1}
\subsection{\subsectitle}

\begin{exampleblock}{\subsectitle}
\begin{verbatim}
P2: x++; // x == 1
P2: z = x; // z == 1
P1: x++; // x == 2
P1: y = x; // y == 2
\end{verbatim}
\end{exampleblock}

\def\subsectitle{Résumé}
\subsection{\subsectitle}

\begin{block}{\subsectitle}
\begin{itemize}
\item Cas 1: x == z == 2;
\item Cas 2: x == 2, z == 1;
\item Cas 3: x == 1, z == 2;
\end{itemize}
\end{block}
\end{columns}

\end{frame}

%%%%%%%%%%%%%%%%%%%%%%%%%%%%%%%%%%%%%%%%%%%%%%%%%%%%%%%%%%%%%%%%%%%%%%%%%%%%%%%%
\def\sectitle{Constatations}
\section{\sectitle}
\def\subsectitle{Inconsistance}
\subsection{\subsectitle}
\begin{frame}{\sectitle}
\begin{block}{\subsectitle}
\begin{itemize}
\item Les exécutions sont inconsistantes
\item Les résultats sont dépendants de l'ordonnanceur
\item Dépendant d'un facteur qui n'est pas contrôlable;
\item On ne peut pas se fier à de tels programmes.
\end{itemize}
\end{block}

\def\subsectitle{Mauvaise vision de ce qui se passe}
\subsection{\subsectitle}
\begin{alertblock}{\subsectitle}
\begin{itemize}
    \item Notons qu'ici on a pris une instruction \texttt{x++}
    \item Cette instruction n'est pas atomique
    \item Une lecture et une écriture
    \item Encore plus de cas différents si l'ordonnanceur interrompt après
    lecture et avant écriture.
\end{itemize}
\end{alertblock}

\end{frame}


%%%%%%%%%%%%%%%%%%%%%%%%%%%%%%%%%%%%%%%%%%%%%%%%%%%%%%%%%%%%%%%%%%%%%%%%%%%%%%%%
\def\sectitle{Section critique}
\section{\sectitle}
\def\subsectitle{Définition}
\subsection{\subsectitle}
\begin{frame}{\sectitle}
    \begin{alertblock}{\subsectitle}
        \begin{itemize}
            \item On définit des zones de programmes pendant lesquelles les
                autres processus ne doivent pas utiliser la ressource partagée.
            \item Le programme en section critique a le contrôle exclusif de la
                ressource.
        \end{itemize}
    \end{alertblock}

    \def\subsectitle{Comment mettre en {\oe}uvre}
    \subsection{\subsectitle}
    \begin{block}{\subsectitle}
        \begin{itemize}
            \item Masquer les interruptions
            \item Utiliser des variables pour signaler un verrouillage
            \item Alternance des processus sur une ressource
            \item Moyens matériels
            \item Sémaphores.
        \end{itemize}
    \end{block}

\end{frame}

%%%%%%%%%%%%%%%%%%%%%%%%%%%%%%%%%%%%%%%%%%%%%%%%%%%%%%%%%%%%%%%%%%%%%%%%%%%%%%%%
\def\sectitle{Solutions}
\section{\sectitle}
\def\subsectitle{Masquage de interruptions}
\subsection{\subsectitle}
\begin{frame}{\sectitle}
\begin{block}{\subsectitle}
\begin{itemize}
\item Masquage des interruptions
\item Le problème ne sera pas coupé
\item Il effectue sa section critique
\item Il réactive les interruptions
\end{itemize}
\end{block}

\def\subsectitle{Problèmes}
\subsection{\subsectitle}
\begin{block}{\subsectitle}
\begin{itemize}
\item Ne fonctionne que pour les architecture mono-processeur;
\item Si le programme fait une erreur, on bloque le système.
\end{itemize}
\end{block}
\end{frame}

\begin{frame}{\sectitle}
    \def\subsectitle{Variable de verrouillage}
    \subsection{\subsectitle}
    \begin{block} {\subsectitle}
        \begin{itemize}
            \item Chaque ressource est associée à un variable qui indique si
                elle est libre ou occupée
            \item Les processus consultent cette variable avant d'utiliser la
                ressource
        \end{itemize}
    \end{block} 

    \def\subsectitle{Problèmes}
    \subsection{\subsectitle}
    \begin{block}{\subsectitle}
        \begin{itemize}
            \item La race condition est déplacée sur la variable de verrouillage
            \item Problème de l'atomicité des opérations, le processus peut-être
                interrompu entre le test de la variable et l'accès à la
                ressource.
        \end{itemize}
    \end{block}
\end{frame}

%%%%%%%%%%%%%%%%%%%%%%%%%%%%%%%%%%%%%%%%%%%%%%%%%%%%%%%%%%%%%%%%%%%%%%%%%%%%%%%%
\begin{frame}[containsverbatim]{\sectitle}
    \def\subsectitle{Code}
    \subsection{\subsectitle}
    \begin{exampleblock}{\subsectitle}
        \begin{verbatim}
while(resource_status == BUSY)
;
resource_status = BUSY;
//Critical section
resource_status = FREE;
        \end{verbatim}
    \end{exampleblock}

    \def\subsectitle{Problèmes}
    \subsection{\subsectitle}
    \begin{alertblock}{\subsectitle}
        \begin{itemize}
            \item Réquisition entre le test \texttt{while} et la ligne suivante
        \end{itemize}
    \end{alertblock}
\end{frame}
%%%%%%%%%%%%%%%%%%%%%%%%%%%%%%%%%%%%%%%%%%%%%%%%%%%%%%%%%%%%%%%%%%%%%%%%%%%%%%%%
\def\sectitle{Test set and Lock}
\section{\subsectitle}
\def\subsectitle{Instruction TSL}
\subsection{\subsectitle}
\begin{frame}[containsverbatim]{\sectitle}
    \begin{block}{\subsectitle}
        \begin{itemize}
            \item Test Set and Lock
            \item L'instruction garantit l'atomicité de l'opération de test et
                d'affectation.
            \item 0: bloqué, 1: libre.
        \end{itemize}
    \end{block}

    \def\subsectitle{Problèmes}
    \subsection{\subsectitle}
    \begin{block}{\subsectitle}
        \begin{itemize}
            \item Perte de temps CPU
            \item Inversion de priorité
            \item Une fois qu'un processus est prêt il doit être élu.
        \end{itemize}
    \end{block}
\end{frame}

%%%%%%%%%%%%%%%%%%%%%%%%%%%%%%%%%%%%%%%%%%%%%%%%%%%%%%%%%%%%%%%%%%%%%%%%%%%%%%%%
\begin{frame}[containsverbatim]{\sectitle}
\begin{exampleblock}{\subsectitle}
\begin{verbatim}
entry:
  tsl reg, flag -- Opération atomique lect. flag + écrit 1
  cmp reg, #0   -- Reg == ancienne valeur de flag
  jnz entry     -- Si non zéro on recommence
  ret           -- Si 0 on a le droit de continuer

quit:
  mov flag, #0  -- On remet 0 dans le flag
  ret
\end{verbatim}
\end{exampleblock}
\end{frame}

%%%%%%%%%%%%%%%%%%%%%%%%%%%%%%%%%%%%%%%%%%%%%%%%%%%%%%%%%%%%%%%%%%%%%%%%%%%%%%%%
\def\subsectitle{Autres méthodes}
\subsection{\subsectitle}
\begin{frame}{\sectitle}
    \begin{block}{\subsectitle}
        \begin{itemize}
            \item Méthode de l'alternance
            \item Algorithme de Dekker
            \item Algorithme de Peterson
        \end{itemize}
    \end{block}
\end{frame}
%%%%%%%%%%%%%%%%%%%%%%%%%%%%%%%%%%%%%%%%%%%%%%%%%%%%%%%%%%%%%%%%%%%%%%%%%%%%%%%%
\def\sectitle{Alternance}
\section{\sectitle}
\begin{frame}[containsverbatim]{\sectitle}
    \def\subsectitle{Alternance version 1}
    \subsection{\subsectitle}
    \begin{exampleblock}{\subsectitle}
        \begin{verbatim}
while(turn != t)
    ;
//Critical section
turn = 1 - t; //1 or 0
\end{verbatim}
\end{exampleblock}

\begin{alertblock}{\subsectitle}
    \begin{itemize}
        \item Si un processus à plus de tâches que l'autre
        \item Si un processus à une plus grande priorité que l'autre
        \item Risque de blocage
    \end{itemize}
\end{alertblock}
\end{frame}

\def\sectitle{Dekker's algorithm}
\section{\sectitle}
\begin{frame}[containsverbatim]{\sectitle}
    \def\subsectitle{Alternance version 2}
    \subsection{\subsectitle}

    \begin{exampleblock}{\subsectitle}
        \begin{verbatim}
(flag[0] = false; flag[1] = false; other = 1 -t;) //init
flag[t] = true;
while(flag[other])
;
//Critical section
flag[t] = false;
        \end{verbatim}
    \end{exampleblock}

    \def\subsectitle{Problèmes}
    \subsection{\subsectitle}
    \begin{alertblock}{\subsectitle}
        \begin{itemize}
            \item Si un processus à plus de tâches que l'autre
            \item Si un processus à une plus grande priorité que l'autre
            \item Risque de blocage
        \end{itemize}
    \end{alertblock}

\end{frame}


%%%%%%%%%%%%%%%%%%%%%%%%%%%%%%%%%%%%%%%%%%%%%%%%%%%%%%%%%%%%%%%%%%%%%%%%%%%%%%%%
\def\sectitle{Peterson}
\section{\sectitle}
\begin{frame}[containsverbatim]{\sectitle}
    \def\subsectitle{Peterson's algorithm}
    \subsection{\subsectitle}
    \begin{exampleblock}{\subsectitle}
        \begin{verbatim}
int other = i -t;
flag[t] = true; // Process interessted in CS
turn = other;
while((flag[other] == true) && (turn == other))
;
//CS
flag[t] = false;
        \end{verbatim}
    \end{exampleblock}


    \def\subsectitle{Problèmes et avantages}
    \subsection{\subsectitle}
    \begin{alertblock}{\subsectitle}
        \begin{itemize}
            \item Si tous intéressés: chacun son tour
            \item Si un processus n'est pas intéressé, il ne fait participe pas
                au choix du processus qui va entrer en section critique;
            \item L'attente est bornée
        \end{itemize}
    \end{alertblock}
\end{frame}

%%%%%%%%%%%%%%%%%%%%%%%%%%%%%%%%%%%%%%%%%%%%%%%%%%%%%%%%%%%%%%%%%%%%%%%%%%%%%%%%
\def\sectitle{Les problèmes classiques}
\section{\sectitle}
\begin{frame}{\sectitle}
    \def\subsectitle{Lecteurs - rédacteurs}
    \subsection{\subsectitle}
    \begin{block}{\subsectitle}
        \begin{itemize}
            \item Producteurs et consommateurs
            \item Des processus partagent une base
            \item Les lecteurs peuvent partager une même données
            \item Les rédacteurs doivent gérer l'accès.
        \end{itemize}
    \end{block}
\end{frame}

%%%%%%%%%%%%%%%%%%%%%%%%%%%%%%%%%%%%%%%%%%%%%%%%%%%%%%%%%%%%%%%%%%%%%%%%%%%%%%%%
\def\sectitle{Producteurs - consommateurs}
\section{\sectitle}
\begin{frame}[containsverbatim]{\sectitle}

    \begin{columns}[t]
        \column{.5\textwidth}
        \def\subsectitle{Producteur}
        \subsection{\subsectitle}
        \begin{exampleblock}{\subsectitle}
            \begin{verbatim}
while(count == SIZE)
;
++count;
buffer[in_curs] = I;
in_curs = (in_curs+1) % SIZE;
            \end{verbatim}
        \end{exampleblock}

        \column{.5\textwidth}
        \def\subsectitle{Consommateur}
        \subsection{\subsectitle}
        \begin{exampleblock}{\subsectitle}
            \begin{verbatim}
while(count == 0)
;
--count;
I = buffer[out_curs];
out_curs = (out_curs+1)%SIZE;
            \end{verbatim}
        \end{exampleblock}
    \end{columns}

\end{frame}


%%%%%%%%%%%%%%%%%%%%%%%%%%%%%%%%%%%%%%%%%%%%%%%%%%%%%%%%%%%%%%%%%%%%%%%%%%%%%%%%
\begin{frame}{\sectitle}
    \begin{block}{\subsectitle}
        \def\subsectitle{Les philosophes}
        \subsection{\subsectitle}
        \begin{itemize}
            \item Les philosophes mangent ou pensent.
            \item Pour manger il a besoin de deux baguettes (une à droite, une à
                gauche);
            \item Comment faire manger les philosophes? (1 philosophe = 1
                processus)
        \end{itemize}
    \end{block}
\end{frame}

%%%%%%%%%%%%%%%%%%%%%%%%%%%%%%%%%%%%%%%%%%%%%%%%%%%%%%%%%%%%%%%%%%%%%%%%%%%%%%%%
\begin{frame}[containsverbatim]{\sectitle}
\def\subsectitle{Les philosohpes - implémentation}
\subsection{\subsectitle}
\begin{exampleblock}{\subsectitle}
\begin{verbatim}
while(true) {
    think();
    get_chopstick(i); // gauche
    get_chopstick(i+1); // droite
    eat();
    release_chopstick(i);
    release_chopstick(i+1);
\end{verbatim}
\end{exampleblock}
\end{frame}


%%%%%%%%%%%%%%%%%%%%%%%%%%%%%%%%%%%%%%%%%%%%%%%%%%%%%%%%%%%%%%%%%%%%%%%%%%%%%%%%
\def\sectitle{Problèmes classiques}
\section{\sectitle}
\begin{frame}{\sectitle}
    \def\subsectitle{Le barbier}
    \subsection{\subsectitle}
    \begin{block}{\subsectitle}
        \begin{itemize}
            \item N chaises, 1 barbier
            \item Le client s'installe s'il y a une chaise libre, sinon il repart
            \item Problème: faire que le barbier serve les client de manière juste.
        \end{itemize}
    \end{block}
\end{frame}


%%%%%%%%%%%%%%%%%%%%%%%%%%%%%%%%%%%%%%%%%%%%%%%%%%%%%%%%%%%%%%%%%%%%%%%%%%%%%%%%
\def\sectitle{Éviter l'attente active}
\section{\sectitle}
\begin{frame}{\sectitle}
\def\subsectitle{Idée}
\subsection{\subsectitle}
\begin{block}{\subsectitle}
\begin{itemize}
    \item Actuellement nous avons de l'attente active (\texttt{while(...);})
    \item Primitives \texttt{sleep()} et \texttt{wakeup()}
    \item Quand on a rien à faire: \texttt{sleep()}
    \item Cela libère le processeur,
    \item Quand on a fini d'utiliser une ressource critique: \texttt{wakeup} 
    permet de réveiller un processus en attente.
\end{itemize}
\end{block}
\end{frame}

%%%%%%%%%%%%%%%%%%%%%%%%%%%%%%%%%%%%%%%%%%%%%%%%%%%%%%%%%%%%%%%%%%%%%%%%%%%%%%%%
\def\sectitle{Éviter l'attente active}
\section{\sectitle}
\begin{frame}[containsverbatim]{\sectitle}
\def\subsectitle{Producteur}
\subsection{\subsectitle}
\begin{columns}[t]
\column{.5\textwidth}
\begin{exampleblock}{\subsectitle}
\begin{verbatim}
while(true) {
    produce(item);
    if(cpt == N) sleep();
    place(item);
    cpt++;
    if(cpt == 1) 
        wakeup(consumer)
}
\end{verbatim}
\end{exampleblock}
\def\subsectitle{Consommateur}
\subsection{\subsectitle}
\column{.5\textwidth}
\begin{exampleblock}{\subsectitle}
\begin{verbatim}
while(true) {
    if(cpt == 0) sleep();
    retrieve(item);
    cpt--;
    if(cpt == N-1) 
        wakeup(producer)
}
\end{verbatim}
\end{exampleblock}
\end{columns}

\end{frame}

%%%%%%%%%%%%%%%%%%%%%%%%%%%%%%%%%%%%%%%%%%%%%%%%%%%%%%%%%%%%%%%%%%%%%%%%%%%%%%%%
\def\sectitle{Éviter l'attente active}
\section{\sectitle}

\begin{frame}{\sectitle}
\def\subsectitle{Problème}
\subsection{\subsectitle}
\begin{block}{\subsectitle}
\begin{itemize}
\item Conflit sur cpt
\item On a toujours un problème si l'ordonnanceur interrompt juste après le
test.
\item Le wakeup est perdu! (Processus pas encore endormi)
\item On veut mémoriser le \texttt{wakeup}: sémaphore.
\end{itemize}
\end{block}
\def\subsectitle{Le sémaphore}
\subsection{\subsectitle}
\begin{alertblock}{\subsectitle}
\begin{itemize}
\item Il faut que l'appel aux sémaphores soient atomiques
\item Non interruptible
\item Nécessite donc le mode système.
\end{itemize}
\end{alertblock}
\end{frame}


%%%%%%%%%%%%%%%%%%%%%%%%%%%%%%%%%%%%%%%%%%%%%%%%%%%%%%%%%%%%%%%%%%%%%%%%%%%%%%%%
\def\sectitle{Sémaphore}
\section{\sectitle}
\begin{frame}{\sectitle}
\def\subsectitle{Description}
\subsection{\subsectitle}
\begin{block}{\subsectitle}
\begin{itemize}
\item Un sémaphore est une variable de contrôle d'accès
\item Il indique le nombre d'éléments disponibles
\item Il tient à jour une liste de processus bloqués qui attendent l'accès à
cette ressources
\item Si la disponibilité de la ressource ne dépasse pas 1, mutex (sémaphore
binaire)
\item >0 : ressources dispo; <0: nombre de processus bloqués.
\item Fonctions P(roberen) et V(erhogen).
\item La file n'est pas nécessairement FIFO.
\end{itemize}
\end{block}
\def\subsectitle{Attention}
\subsection{\subsectitle}
\begin{alertblock}{\subsectitle}
\begin{itemize}
\item On initialise les sémaphores hors des processus à sections critiques.
\end{itemize}
\end{alertblock}
\end{frame}

%%%%%%%%%%%%%%%%%%%%%%%%%%%%%%%%%%%%%%%%%%%%%%%%%%%%%%%%%%%%%%%%%%%%%%%%%%%%%%%%
\def\sectitle{Sémaphore}
\section{\sectitle}
\begin{frame}{\sectitle}
\def\subsectitle{Producteurs - consommateurs}
\subsection{\subsectitle}
\begin{block}{\subsectitle}
\begin{itemize}
    \item Un mutex pour contrôler les accès critiques au buffer
    \item Un sémaphore pour la disponibilité en consommation (init: 0;
    \texttt{filled})
    \item Un sémaphore pour la disponibilité en production (init: N;
    \texttt{empty})
\end{itemize}
\end{block}

\begin{alertblock}{\subsectitle}
\begin{itemize}
    \item Tous les processus utilisent le mutex
    \item Les consommateurs P sur \texttt{filled} et V sur \texttt{empty}
    (consomment une case occupée et créent une case vide)
    \item Les producteurs P sur \texttt{empty} et V sur \texttt{filled}
    (consomment une case vide et créent une case occupée)
\end{itemize}
\end{alertblock}
\end{frame}

\def\sectitle{Sémaphore}
\section{\sectitle}
\begin{frame}[containsverbatim]{\sectitle}
\begin{columns}[t]

\column{.5\textwidth}
\def\subsectitle{Producer}
\subsection{\subsectitle}
\begin{exampleblock}{\subsectitle}
\begin{verbatim}
P(empty)
P(acces)
add(item)
V(access)
V(filled)
\end{verbatim}
\end{exampleblock}

\column{.5\textwidth}
\def\subsectitle{Consumer}
\subsection{\subsectitle}
\begin{exampleblock}{\subsectitle}
\begin{verbatim}
P(filled)
P(acces)
add(item)
V(access)
V(empty)
\end{verbatim}
\end{exampleblock}
\end{columns}
\end{frame}

%%%%%%%%%%%%%%%%%%%%%%%%%%%%%%%%%%%%%%%%%%%%%%%%%%%%%%%%%%%%%%%%%%%%%%%%%%%%%%%%
\def\sectitle{Moniteurs}
\section{\sectitle}
\begin{frame}{\sectitle}
\begin{block}{\subsectitle}
\begin{itemize}
\item Primitive de haut niveau
\item Procédures, variables, structures de données
\item Un seul processus actif dans un moniteur (sinon il est suspendu)
\item Accès externes aux variables interdit.
\item \texttt{wait/signal}
\end{itemize}
\end{block}

\begin{block}{\subsectitle}
\begin{itemize}
\item \texttt{signal(i)} réveil un processus qui attend sur \texttt{i}
\end{itemize}
\end{block}
\end{frame}

%%%%%%%%%%%%%%%%%%%%%%%%%%%%%%%%%%%%%%%%%%%%%%%%%%%%%%%%%%%%%%%%%%%%%%%%%%%%%%%%
\def\sectitle{Moniteurs}
\section{\sectitle}
\begin{frame}{\sectitle}
\begin{block}{\subsectitle}
\begin{itemize}
    \item On peut remplacer notre schéma producteur consommateur avec le 
    sémaphore \texttt{access} en moins.
    \item Dépendant du langage : géré par le compilateur.
\end{itemize}
\end{block}
\end{frame}

\end{document}
